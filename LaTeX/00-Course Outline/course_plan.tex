\documentclass[a4paper, 12pt]{article}
\usepackage[top=1in,bottom=1in,left=3.2cm,right=2.6cm]{geometry}
\usepackage[hidelinks]{hyperref}
\title{Introduction to {\LaTeX} for Researchers} 
\author{\LARGE Jubayer Hossain} 


\begin{document}
% Title Page 
\maketitle 
\section*{\LARGE Course Outline and Resources}


\section*{Install \LaTeX} 
\begin{itemize}
	\item Windows: \url{https://miktex.org/}
	\item Linux: sudo apt install texlive-latex-extra
\end{itemize}


\section*{{\LaTeX} Editor -- Online}
\begin{itemize}
	\item Sharelatex/Overleaf: \url{https://www.overleaf.com/}
	\item Latexbae: 
	\url{https://latexbase.com/d/1648f17f-f52f-4653-8bca-7c9066c9a8de}
	\item CoCalc: \url{https://cocalc.com/doc/latex-editor.html}
\end{itemize}


\section*{{\LaTeX} Editor -- Offline}
\begin{itemize}
	\item TexStudio: \url{https://www.texstudio.org/}
	\item Texmaker: \url{https://www.xm1math.net/texmaker/}
\end{itemize}


\section*{Resources} 
\begin{itemize}
	\item \url{https://www.latex-project.org/help/documentation/}
	\item \url{https://www.overleaf.com/learn}
	\item \url{https://www.latex-tutorial.com/}
	\item \url{https://www.latextemplates.com/} 
	\item \url{https://en.wikibooks.org/wiki/LaTeX/Mathematics}
\end{itemize}


\section*{Section-1: Introduction} 
\begin{itemize}
	\item Overview 
	\item Install \LaTeX 
	\item Create a First {\LaTeX} Document 
	\item Section Levels and ToC(Table of Contents)
	\item Comments 
\end{itemize}


\section*{Before Section-2: Package Vs Class}
\subsection*{Class} 
Class files have .cls extension and are responsible for the overall layout, 
structuring (organizing), and formatting of a document. For example, a book 
class contains chapters, sections, parts; while an article contains sections, 
subsections. Each of which has a different set and style of formatting. The 
class files will be available on your system as article.cls, book.cls, etc..

\subsection*{Packages} 
Packages have a .sty extension also called as style files. The primary 
objective of the package is to add some functionality, irrespective of the 
class in which it will be used. Some examples to note are, 'geometry' package, 
that is used to specify margins and paper size; 'graphicx' package, that is 
used to include an image, etc.. All of these can be used in any class. The 
packages will be available on your system as geometry.sty, graphicx.sty, etc.. 
The coming sessions make use of the packages.


\section*{Section-2: Formatting Content} 
\begin{itemize}
	\item Formatting Text 
	\item Coloring Text   
	\item Aligning Text 
	\item Spacing Text 
	\item Unordered and Ordered List(Bullets and Numbering) 
\end{itemize}


\section*{Section-3: Inserting Images} 
\begin{itemize}
	\item Introduction 
	\item Set the folder path 
	\item Inserting Image 
	\item Captioning, labelling and referencing
\end{itemize}


\section*{Section-4: Creating Tables} 
\begin{itemize}
	\item Introduction 
	\item Creating Tables 
	\item Table Boarders  
	\item Captioning, labelling and referencing
\end{itemize}

\section*{Section-5: More on Table and Images} 
\begin{itemize}
	\item Image Properties 
	\item Image Positioning
	\item Image Boarders 
	\item Table Positioning
	\item Figures and Sub-Figures 
	\item Tables and Sub-tables 
	\item Merging Row and Columns 
	\item Handling Large Tables 
\end{itemize}

 
\section*{Section-6: Styling Pages} 
\begin{itemize}
	\item Paper Size: \url{https://en.wikipedia.org/wiki/Paper_size}
	\item Paper Size and Margins  
	\item Page Styles (Header and Footer)
	\item More on Formatting (Footnotes, Orientation, Page Break) 
	\item Multicolumn Document(multicol package)
\end{itemize}



\section*{Section-7: Math,Statistics and Chemical Equation} 
\begin{itemize}
	\item Math Environment 
	\item Equations 
	\item Theorem
	\item Matrix 
	\item Statistics 
	\item Chemical Equations  
\end{itemize}

\section*{Section-8: Referencing and Indexing} 
\begin{itemize}
	\item Creating Title Page 
	\item Hyperlinks 
	\item Cross Referencing
	\item Creating Index 
	\item Creating Bibliography
\end{itemize}


\section*{Section-9: Presentation using Beamer} 
\begin{itemize}
	\item Introduction to Beamer 
	\item Blocks and Columns 
	\item Overlays 
	\item Customize Themes 
	\item Aspect Ratio 
	\item Recapitulate 
\end{itemize}


\section*{Section-10: Projects} 
\begin{itemize}
	\item Article Writing 
	\item Research Paper  
	\item Report / Assignment 
	\item Thesis 
\end{itemize}

\end{document}
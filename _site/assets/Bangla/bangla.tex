\documentclass{article}
\usepackage[banglamainfont=Kalpurush, 
banglattfont=Siyam Rupali
]{latexbangla}
\begin{document}
	পিথাগোরাস(Pythagoras)-এর উপপাদ্যটি হল, \textit{সমকোণী ত্রিভুজের অতিভুজের উপর অঙ্কিত  
		বর্গক্ষেত্রের ক্ষেত্রফল অপর দুই বাহুর উপর অঙ্কিত বর্গক্ষেত্রের ক্ষেত্রফলের সমষ্টির সমান।}
	অর্থাৎ কোন সমকোণী ত্রিভুজের অতিভুজ $c$ এবং অপর দুই বাহু $a$ এবং $b$ হলে,
	\[c^2=a^2+b^2\]
	লক্ষ্য করুন, এখন পর্যন্ত টেক্সট প্রদর্শনের জন্য \textbf{কালপুরুষ} ফন্ট ব্যবহৃত হয়েছে।\\
	\texttt{এবার, টেলিটাইপ(Teletype) টেক্সট প্রদর্শনের জন্য \textbf{সিয়াম রূপালী} 
		ফন্ট ব্যবহৃত হল।}\\
	পুনরায় টেক্সট প্রদর্শনের জন্য \textbf{কালপুরুষ} ফন্ট ব্যবহৃত হচ্ছে।
\end{document} 